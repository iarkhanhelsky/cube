\documentclass[a4paper,11pt]{article}
\usepackage[T2A]{fontenc}     
\usepackage[utf8]{inputenc} 
\usepackage{amsmath}
\usepackage{hyperref}
\usepackage{geometry} % Меняем поля страницы
\geometry{left=2cm}% левое поле
\geometry{right=1.5cm}% правое поле
\geometry{top=1cm}% верхнее поле
\geometry{bottom=2cm}% нижнее поле

\renewcommand\contentsname{Содержание}
\renewcommand\partname{ }
\renewcommand{\thepart}{\arabic{part}}

\title{Пояснительная записка к проекту "Color Cube"}
\author{Архангельский Илья}


\begin{document}
  \begin{titlepage}
   \begin{center}
		БЕЛОРУССКИЙ ГОСУДАРСТВЕННЫЙ УНИВЕРСИТЕТ \\
		ФАКУЛЬТЕТ ПРИКЛАДНОЙ МАТЕМАТИКИ И ИНФОРМАТИКИ
	\end{center}
	\vspace{10em}
	\begin{center}
		\LARGE {Пояснительная записка к проекту "Color Cube"}
		\linebreak	 
		
    
	\end{center}
	\vspace{3em}
	\begin{flushright}
	  
	
 	Автор: \\	Архангельский И.А. \\ 
 	
 	  \vspace{1em}
 	
 	  
 	
	\end{flushright}
	
	\vfill
	\begin{center}
		Минск, 2012
	\end{center}
  \end{titlepage}
  \tableofcontents
  \newpage
  \part{Соответсвие техническому заданию}
  \begin{itemize}
    \item Возможности менять положение наблюдателя 
    \newline
     Вращение куба фактически, есть изменение положения наблюдателя вокруг куба. Изменение размера куба, есть изменение расстояния между наблюдателем и кубом.
    \item Вращать куб  
        \newline
      Реализовано
    \item Менять размер куба 
        \newline
      Реализовано 
    \item  задавать шаг (в примере 16x16x16, а надо бы и 8x8x8) 
        \newline
      Реализовано изменение количества слоев в диапазоне $[2:20]$, отрисовка с большего количества слоев требует более оптимальных алгоритмов отрисовки(Z-buffer, двойная буфферизация), и использования видеокарты напрямую, что ограничевает кроссплатформенность приложения. 
    \item включать отключать вывод рёбер.  
        \newline
        
  \end{itemize} 
  
  Также реализовано
  \begin{itemize}
    \item Затухание вращения
  \end{itemize}
  \newpage
  \part{Коментарии и пояснения по реализации}
  Куб задается 8 точками в пространстве (вершины). Центр куба находится в точке $(0,0,0)$. Грани строятся по 4 точкам.
  \section{Вращение}
  Вращение куба реализовано с при помощи преобразования координат. Если рассматривать вершину куба как вектор, то поворот в некоторой плоскости есть умножение вектора на специальную матрицу.
  
  \subsection*{Rolling. Вращение вокруг оси $x$}
  Матрица поворота: $M_x(\alpha) =
  \begin{pmatrix}    
1 &   0           & 0           \\
0 & \cos \alpha   &  -\sin \alpha \\
0 & \sin \alpha & \cos \alpha 
 \end{pmatrix}$
 
  \subsection*{Pitching. Вращение вокруг оси $y$}
Матрица поворота  $M_y(\alpha) = 
\begin{pmatrix} 
\cos \alpha   & 0 & \sin \alpha \\
   0          & 1 &  0          \\
 -\sin \alpha & 0 & \cos \alpha
\end{pmatrix} $
  \subsection*{Yawing. Вращение вокруг оси $z$}
Матрица поворота:  $M_z(\alpha) =
\begin{pmatrix} 
\cos  \alpha  &  -\sin \alpha & 0 \\
\sin \alpha & \cos \alpha & 0 \\
   0          & 0           & 1
\end{pmatrix} $
  \section{Проецирование}
 Для проецирования было решено использовать диметрическую проекцию. То есть $x$ и $y$ координаты искажаются. Формула искажения была выбрана такая : 
\[  x' = x\frac{d-z}{d+a} \]
где, 


 
\textbf{d} - это некоторая константа, фактически отвечающая за расстояние от объекта до плоскости экрана. 



\textbf{a} - длинна ребра куба. 
\\
В текущей реализации $d = 8a$
Точно такой же коэффициент используется для преобразования 
\[  y' = y\frac{d-z}{d+a} \]
  
  \section{Изменение длины ребра}
  Так как куб отпозиционирован так, что его центр находится в точке $(0,0,0)$, 
  для изменения длины ребра куба, достаточно изменить длинну векторов, которые соответсвуют вершинам куба. Тем самым к каждой вершине применяем преобразование
  \linebreak
\[ 
\begin{cases}
   x' = x\frac{l^{\frac{1}{3}}}{\sqrt{x^2+y^2+z^2}} \\
   y' = y\frac{l^{\frac{1}{3}}}{\sqrt{x^2+y^2+z^2}} \\
   z' = z\frac{l^{\frac{1}{3}}}{\sqrt{x^2+y^2+z^2}}
\end{cases}
 \] 
  \section{Разбиение на сегменты}
  \section{Отрисовка}
\part{Ссылки}
  \begin{itemize}
    \item \href{https://github.com/DziedMaroz/Cube/tree/ColorCube/src/Cube}{Последняя ревизия исходных текстов проекта}
     
  \end{itemize}
\end{document}
